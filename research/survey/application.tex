\section{Applications}
In this section, we first show a list of possible applications and discuss several representative applications in detail. We will see from those examples that submodularity is such a natural property  that many real-world problems can be cast in to the framework of submodular optimization (maximization).


\subsection{List of Possible Applications}
\begin{itemize}
\item Combinatorial Problems: set cover, max $k$ coverage, vertex cover, edge cover, graph cut problems etc.
\item Networks: social networks, viral marketing, diffusion networks etc.
\item Graphical Models: image segmentation, tree distributions, factors etc.
\item NLP: document summarization, web search, information retrieval
\item Machine Learning: active/semi-supervised learning etc.
\item Economics: markets, economies of scale
\end{itemize}



\subsection{Classical Problems Revisited}
We first show that several well-known problems actually fit into our standard submodular maximization framework.

\paragraph{Exemplar Based Clustering}
Clustering is one of the most important tasks in the area of data mining. In the k-medoid problem \cite{KR09} one tries to minimize the sum of pairwise dissimilarities/distances between exemplars and the elements of the dataset. Let $d: V \times V \rightarrow \bbR^+\cup\{0\}$ be a function that measures the pairwise dissimilarity, we define the k-medoid loss function as following,
$$L(S) = \sum_{e\in V} \min_{v\in S} d(e, v).$$

It is quite straightforward (by Property \ref{prop:addition}, \ref{prop:max}) to show that $-L(S)$ is submodular. By introducing an auxiliary element $e_0$, we can transform $L$ into a non-negative monotone submodular function,
$$f(S) = L(\{e_0\}) - L(S \cup \{e_0\}).$$ 

A k-medoid problem can then be formulated as a submodular maximization problem subject to a cardinality constraint,

$$\argmax_{S\subseteq V: |S| \leq k} f(S).$$


\paragraph{Set Cover Problem}
The \emph{set cover problem} is an important problem in combinatorial optimization where we are given a collection of subsets of a set $E$, i.e. $V = \{C_1, C_2, \ldots, C_n\}$ where each $C_i \subseteq E$. We define a function $f:2^V \rightarrow \bbR$ such that $f(S) = |\cup_{C\in S} C|$. We can interpret $f$ as follow: given $S$ as a subset of $V$, the value of $f(S)$ is the number of distinct elements covered by the sets in $S$.

One can easily verify that $f$ satisfies the diminishing return property thus is a submodular function. Furthermore, it is clear that $f$ is non-decreasing.  Now given the cardinality constraint, we want to solve the following,

$$\argmax_{S\subseteq V: |S|\leq k} f(S).$$

We may also assign each $C\in V$ a non-negative cost $w(C)$ (e.g. the size of $C$), and given a total budget $W$, our goal is to find a solution of the following,
$$\argmax_{S\subseteq V: w(S) \leq W} f(S),$$
where $w(S) = \sum_{C\in S} w(C)$. This is a monotone submodular maximization problem under the knapsack constraint.



\paragraph{Maximum Spanning Forest}
Let us consider a graph $G = (V, E)$ where $V$ is the set of vertices and $E$ is the set of edges. In this case we consider $E$ as the ground set and define 
$$\calI = \{S\subseteq E ~|~\text{edge-induced graph}~G(V, S)~\text{does not contain a circle}\}.$$

One can verify (via definition) that $(E, \calI)$ is a matriod. The rank function of $(E, \calI)$ can be interpreted as the size of the maximum spanning forest of an edge-induced graph, i.e. given $S\subseteq E$, $r(S)$ is the size of maximum spanning forest (in terms of number of edges) of $G(V, S)$. 




Now assume that we assign each $e\in E$ a weight $w_e \geq 0$. Let $f: 2^E\rightarrow \bbR$ with $f(S) = \sum_{e\in S}w_e$ be the objective function we want to maximize. Clearly $f$ is monotone and (sub)modular. we consider the following optimization problem,
$$\argmax_{S \in \calI} f(S).$$
This is exactly the \emph{Maximum Spanning Forest} problem and by Theorem \ref{thm:matroid}, we can solve it efficiently (and exactly) using Algorithm \ref{algo:greedy}.


\paragraph{Maximum Cut in Graphs}
Given an undirected graph $G = (V, E)$ and a non-negative capacity function $c: E \rightarrow \bbR^+\cup\{0\}$, the cut capacity function $f:2^V \rightarrow \bbR$ defined by $f(S) = \sum_{e\in \delta(S)} c(e)$ is submodular, where $\delta(S) = \{e\in E~|~e~\text{has exactly one vertex in}~S\}$ i.e. the set of edges crossing $S$ and $E\backslash S$. To shows why $f$ is submodular, we introduce an auxiliary function $f_e: 2^{\{u, v\}} \rightarrow \bbR$ for each $e = \{u, v\}\in E$ which is defined on the graph $G_e = (\{u,v\}, \{e\})$. We define, 
\begin{itemize}
\item $f_e(\{u, v\}) = f_e(\emptyset) = 0$
\item $f_e(\{u\}) = f_e(\{v\}) = w_e$
\end{itemize}
Then $f_e$ is submodular. We have,
$$f(S) = \sum_{e\in E} f_e(S \cap \{u, v\}).$$
The submodularity of $f$ follows Property \ref{prop:addition} and Property \ref{prop:restriction}.

An interesting optimization problem can then be formulated as following,
$$\argmax_{S\subseteq E: |S| \leq k} f(S).$$





\subsection{Applications to NLP}
\paragraph{Summarization}
In the task of summarization, we are given a ground set $V$ and we want to find a subset of $V$ which maximizes some quality measurement under certain constraints. One popular formulation is that, given a function $f: 2^V \rightarrow \bbR$ that measures the quality of a summarization, we try to solve,
$$\argmax_{S\in\calI} f(S)$$
where $\calI$ is a knapsack constraint.

Lin et al. \cite{LB11} pointed out that a lot of existing work for document summarization task fit the knapsack optimization framework. Furthermore the quality measurement functions being used are usually submodular. Lin et al. also proposed a class of submodular functions that outperform previous work in many aspects. Each of those functions combines two terms, one that encourages the summary to be representative of the corpus, and the other positively rewards diversity. In the task of speech summarization, several submodular functions were discussed by Lin \cite{L12}. More discussion on the applications of submodular optimization to summarization can be found in \cite{L12}.

\paragraph{Word Alignment}
Word alignment is a key component in most statistical machine translation systems. Unlike classical approaches that utilize graphical models, Lin et al. \cite{LB11word} viewed word alignment problem as submodular maximization problem under matroid constraints. 

Suppose that we are given a source language (English) string $e_1, e_2, \ldots, e_n$ and a target language (French) string $f_1, f_2, \ldots, f_m$. Let $V = \{(i, j) ~|~ i\in[n], j\in[m] \}$ be the ground set. The goal is to find a match $S\subseteq V$ that maximizes a certain quality function under some constraints. 

Let $P_1, \ldots, P_m$ be a partition of $V$ where $P_j = [n]\times \{j\}$. The \emph{fertility} restriction of a word $f_j$ then requires that $f_j$ can only match at most $k_j$ words in $e_1,\ldots, e_n$, or equivalently the match $S \subseteq V$ satisfies $|S \cap P_j| \leq k_i$ for all $j\in[m]$. Such a constraint is called \emph{partition constraint}, which is an important instance of matroid. Lin et al. then proposed a submodular function as the quality function, which is a composition of a concave function and a modular function. We refer readers to \cite{LB11word} for details.





\subsection{Applications to Social Networks}
\paragraph{Influence Maximization}
Domingos and Richardson \cite{DR01,RD02} posed a fundamental algorithmic problem: if we can try to convince a subset of individuals (at most $k$) to adopt a new product or innovation, and the goal is to trigger a large cascade of further adoptions, which set of individuals should we target? 
Kempe et al. \cite{KKT03} considered such question as a discrete optimization problem. Let $G = (V, E, W)$ a social network (a directed graph with non-negative weights for each edge), two basic models of influence spreading were considered in \cite{KKT03},
\begin{itemize}
\item Linear Threshold Mode: after initializing $A \subset V$ as the set of active nodes, at each time step, a node $v$ will turn from inactive to active if the total weights of its active neighbors is above some threshold $\theta_v$. Here $\theta_v$ may or may not be a random variable.
\item Independent Cascade Model:  after initializing the active set $A \subset V$, at each time step each active node $v$ will activate its neighbor $u$ (for all inactive neighbors) with probability $w_{v,u}$ (which is the parameter of this network).
\end{itemize}

Now let $\sigma(A)$ be the expected number of active nodes after $T$ time steps ($T$ is usually set to be large). \cite{KKT03} proved that for both models, $\sigma: 2^V \rightarrow \bbZ$ is a monotone submodular function.


In our application, one is unlikely to able to efficiently compute the exact value of $\sigma(A)$. In fact, one can extend the result of Theorem \ref{thm:1978} to show that, by using  $(1 + O(\eps))$-approximate values for the function to be optimized, we obtain a $(1 - 1/e - \eps)$-approximation of the optimum.


\paragraph{Network Structure Inference}
Gomez Rodriguez et al. \cite{GLK10} first introduced submodular maximization to the context of network structure learning. The consider the problem of learning the network structure in a influence network.


\subsection{Applications to Machine Learning}

\subsection{More Applications}



\begin{itemize}
\item \cite{MKC+15} distributed k-centers using submodular optimization.
\end{itemize}

