\section{Introduction}
Submodularity is a property of set functions with deep theoretical and practical consequences. Submodular functions occur in a variety of applications, including representative skyline selection \cite{SLN+11}, network structure learning \cite{GLK10}, influence maximization \cite{KKT03}, document summarization \cite{LB11}, image segmentation \cite{BJM01,KKT09} and many others.


There is a large body of research in submodular optimization, and our main focus of this survey is submodular maximization. In particular, we will cover the recent advances of maximizing submodular functions in distributed and streaming setting. Part of the reason we do not discuss submodular minimization is that, one can convert a minimization problem to a convex optimization problem through the Lov{\'a}sz' extension \cite{L83}, and it can be therefore solved in polynomial time. On the other hand, most submodular maximization problems we will discuss are NP-Hard, and can only be solved efficiently via approximation algorithms.


\paragraph{Roadmap:}
This survey is organized as follows. In Section \ref{sec:submodularity} we introduce several important properties of submodular functions; we then discuss different set systems that serve as constraints in submodular optimization problems; we also cover several classical results for submodular maximization in RAM model. In Section \ref{sec:applications} we discuss the applications of submodular maximization, including both classical problems and formulations appear in recent research. We then cover recent progress made in the area of streaming/distributed submodular maximization in Section \ref{sec:streaming} and Section \ref{sec:distributed}.  We conclude this survey by discussing possible further research in Section \ref{sec:conclusion}.






\paragraph{Notations:}
Through out this survey, we use $V$ or $E$ to represent the ground set we consider; $2^V$ is the power set of $V$ (i.e. the set of all subsets of $V$); in general, $A^V$ is the collection of maps from $V$ to $A$; for simplicity, we sometime write $A \cup \{x\}$ as $A + x$. Other notation will be defined later when it is first introduced.
