\section{Streaming Submodular Maximization}
In this section, we cover recent results for submodular maximization over data streaming. In the streaming model, we consider the ground set $V$ as an ordered sequence of items $e_1, e_2, \ldots, e_n$. We process the items one by one and the maximum space being used should be sublinear (i.e. $o(n)$).


Unlike in centralized setting where one can always calculate the function value, in streaming model we need to assume value oracle and membership oracle explicitly.  Most algorithms developed for centralized submodular maximization require access to all items in the ground set and hence can not be directly applied in streaming setting. In general streaming submodular maximization is harder and was considered only very recently. We provide a table for best known approximation bounds in Table \ref{table:streaming} and review some technical details of those related papers.

\begin{table}[t]
\centering
\begin{tabular}{|l|l|l|}
\hline
constraint & monotone  &  non-negative \\
\hline
cardinality & $\frac{1-\eps}{2}$ \cite{BMK+14} & $\frac{1 - \eps}{2 + e}$ \cite{CGQ15}, R\\
\hline
matroid & $1/4$ \cite{CK14} & $\frac{1 - \eps}{4 + e}$ \cite{CGQ15},  R \\
\hline
matching & $4/31$ \cite{CK14} & $\frac{1 - \eps}{12 + \eps}$ \cite{CGQ15}, R \\
\hline
intersection of $p$ matroids & $\frac{1}{4p}$ \cite{CK14} & $\frac{(1 - \eps)(p - 1)}{5p^2 -4p + \eps}$ \cite{CGQ15}, R\\
\hline
$p$-matchoid & $\frac{1}{4p}$ \cite{CGQ15} & $\frac{(1-\eps)(2-o(1))}{(8+e)p}$ \cite{CGQ15}, R\\
\hline
\end{tabular}
\caption{Best known approximation bounds for submodular maximization in streaming model. Bounds for randomized algorithms that hold in expectation are marked (R).}
\label{table:streaming}
\end{table}






\subsection{Results for  Cardinality Constraint}
Monotone submodular maximization with cardinality constraint is the simplest setting and it is most likely to have a good solution. Krause et al. \cite{KG10} gave the first discussion on maximizing a monotone submodular function over data streams. Their approach makes strong assumption on the input data and the update time per item is as high as $O(k)$ value queries ($k$ is the size of the solution).  Kumar et al. \cite{KMV+15} proposed a single-pass which assumes an upper bound on $\max_{e\in V}\Delta(e|\emptyset)$. \cite{KMV+15} also includes a multi-pass algorithm with space usage depending on the ground set. Badanidiyuru et al. \cite{BMK+14} provided a nice solution to monotone submodular maximization under cardinality constraint. They also discussed the situation when value oracle is not realistic in streaming model. 

As for non-monotone submodular maximization, Buchbinder et al. \cite{BFS15} gaves a $.0893$-approximation, and the very general framework of Chekuri et al. \cite{CGQ15} yields a $\frac{1-\eps}{2 + e}$-approximation (in expectation). 

\chensays{add a table for streaming results}
\paragraph{Monotone Case:} We only describe \cite{BMK+14}, which is the best known result in many aspects.  In Algorithm \ref{algo:greedy}, it is known that the marginal gain $\Delta(e_{i+1}|S_i)$ of the next element $e_{i+1}$ added is at least $(\opt - f(S_i))/k$ where $\opt=\max_{S:|S|\leq k} f(S)$, and $S_i$ is the set of the first $i$ elements picked. This intuition also helps in streaming setting: if we know $\opt$ in advance, we can use $\frac{\opt/2 - f(S)}{k - |S|}$ as a  threshold of marginal gain to decide if a new element $e$ should be added to the solution $S$ or not (denote such an algorithm as \knowopt). Of course we do not know $\opt$, but for any $e\in V$, we must have $m \leq \opt \leq k\cdot m$ where $m = \max_{e\in V}f(\{e\})$. So we can guess the value of $\opt$ using values in $\{(1 + \eps)^i~|~i\in\bbZ, m\leq (1 + \eps)^i \leq k\cdot m\}$ and for each guess we run an instance of \knowopt. Such an algorithm, however, requires two passes with $m$ being decided in the first pass. With the trick of lazy-evaluation, we can actually estimate $m$ on the fly. The final algorithm can is depicted in Algorithm \ref{algo:sieve}.

\begin{algorithm}[H]
\DontPrintSemicolon % Some LaTeX compilers require you to use \dontprintsemicolon instead
\KwIn{$V$ as data stream, $f$ a monotone submodular function, $k$ the size constraint, $\eps$ a parameter}
\KwOut{a set $S \subseteq V$}
$O = \{(1 + \eps)^i~|~i\in \bbZ\}$\;
\tcc*[h]{maintain the sets only for the necessary $v$'s lazily}\;
For each $v\in O, ~S_v \gets \emptyset$\;
$m \gets 0$\;

\For{each $e$ in the data stream} {
  m $\gets \max\{m, f(\{e\})\}$\;
  $O\gets \{(1 + \eps)^i~|~m \leq (1 + \eps)^i \leq 2\cdot k \cdot m\}$\;
  Delete all $S_v$ such that $v \in O$\;
  \For{$v \in O$}{
    \If{$\Delta(e|S_v) \geq \frac{v/2 - f(S_v)}{k - |S_v|}$ and $|S_v|<k$}{
      $S_v \gets S_v \cup \{e\}$\;
    }
  }
}
\Return{$\argmax_{S_v: v\in O}f(S_v)$}\;
\caption{{\sc Sieve-Streaming} for submodular maximization}
\label{algo:sieve}
\end{algorithm}

It can be shown that Algorithm \ref{algo:sieve} achieves $(1/2-\eps)$-approximation by using $O(\frac{k \log k}{\eps})$ space and $O(\frac{\log k}{\eps})$ update time per item.


\paragraph{Non-negative Non-Monotone Case:} Non-negative non-monotone case can be much harder. The state of the art comes from \cite{CGQ15}. The algorithm we discuss here is an instance of their general framework and it goes as follows: the algorithm is parameterized with a threshold $\alpha > 0$ and  maintains a buffer $B$ of size $\frac{k}{\eps}$; the algorithm then keeps every new coming item with marginal gain above the threshold into the buffer, until the buffer is overflowed; an item $e$ is then randomly sampled from $B$ and added to the solution $S$; all other items with marginal gain below the threshold will be removed from $B$.   We describe the detail in Algorithm \ref{algo:random-streaming}.

\begin{algorithm}[H]
\DontPrintSemicolon % Some LaTeX compilers require you to use \dontprintsemicolon instead
\KwIn{$V$ as data stream, $f$ a non-negative submodular function, $k$ the size constraint, $\eps$ a parameter}
\KwOut{a set $S \subseteq V$}
$B\gets \emptyset, S\gets \emptyset$\;
\For{each $e$ in the data stream} {
  \If{$|S| < k$ and $\Delta(e|S) > \alpha$}{
    $B \gets B + e$\;
  }
  \If{$|B| > \frac{k}{\eps}$}{
    $e \gets $ uniformly random from $B$\;
    $B \gets B - e, S \gets S + e$\;
    \For{all $e'\in B$ s.t. $\Delta(e'|S)\leq \alpha$}{
      $B \gets B - e'$\;
    }
  }
}
$S' \gets$ unconstrained offline algorithm on $B$\;
\Return{$\argmax_{A\in\{S, S'\}}f(A)$}\;
\caption{{\sc Random-Streaming} for non-monotone submodular maximization}
\label{algo:random-streaming}
\end{algorithm}

When taking $\frac{1 - \eps}{(2 + \eps)k}\opt\leq \alpha \leq \frac{1 + \eps}{(2 + e)k}\opt$, Algorithm \ref{algo:random-streaming} yields a $\frac{1 - \eps}{2 + \eps}$-approximation. Of course we do not know $\opt$, but the same trick in {\sc Sieve-Streaming} also applies here. The space usage is bounded by $O(\frac{k\log k}{\eps})$.



\subsection{Results for Matroids and Its Intersection}
For general non-negative submodular maximization subject to matroid constraint, the general framework in \cite{CGQ15} yields a $\frac{1 - \eps}{4 + \eps}$-approximation (which we will not discuss in detail). For monotone case, a better result was given by Chakrabarti and Kale \cite{CK14}, which follows the algorithm of Varadaraja \cite{V11} on modular maximization. In this section, we mainly discuss the results from \cite{V11,CK14}.

Varadaraja \cite{V11} proposed an algorithm for normalized modular maximization subject to the intersection of $p$ matroids (it actually works for more general problems, be we only need its result for modular function). The algorithm is very simple and we would like to describe here as pseudocode. Also recall that, for a normalized modular function $f$, we have $f(S) = \sum_{e\in S}f(\{e\})$. 



\begin{algorithm}[H]
\DontPrintSemicolon % Some LaTeX compilers require you to use \dontprintsemicolon instead
\KwIn{$V$ as data stream, $f$ a non-decreasing  modular function, $p$ matroid constraints $\calI_1, \ldots, \calI_p$, $\gamma > 0$ is a parameter}
\KwOut{a set $S \subseteq V$}

\For{each $e$ in the data stream} {
  \If{$S+e \in \cap_{i}\calI_i$}{
    $S \gets S + e$\;
  }
  \Else{
    $C \gets \emptyset$\;
    \For{each $\calI_i$}{
      \If{$S + e \not\in \calI_i$}{
        $e_i \gets \argmin_{e': S+e-e' \in \calI_i}f(\{e'\})$\; 
        $C \gets C \cup e_i$\;
      }
      \If{$f(\{e\}) \geq (1 + \gamma)\cdot \left(\sum_{e'\in C}f(\{e'\}) \right)$}{
        $S \gets S \cup \{e\} \backslash C$\;
      }
    }
  }
}
\Return $S$\;
\caption{Streaming algorithm for non-decreasing modular function}
\label{algo:streaming-modular}
\end{algorithm}

Varadaraja proved that Algorithm \ref{algo:streaming-modular} returns a $\frac{\gamma}{(p\cdot (1 + \gamma) - 1)(1 + \gamma)}$-approximation. In particular, this algorithm returns a $(1 + \gamma)$-approximation for matroid constraint. For convenience, we also call the non-decreasing normalized modular maximization
problem as \emph{maximum weighted} (MW) problem. 

Chakrabarti et al. \cite{CK14} proposed a general framework called \emph{compliant} algorithm. A key theorem in \cite{CK14} then shows that if a compliant algorithm for MW has some approximation guarantee, running the same algorithm for normalized monotone submodular also gives certain guarantee, informally,
\begin{theorem}
  \label{thm:compliant}
  If a compliant algorithm for MW runs on parameter $\gamma$ and returns $C_{\gamma}$-approximation, running the same algorithm for normalized monotone submodular function gives $(C_{\gamma} + 1 + 1/\gamma)$-approximation.
\end{theorem}

As expected, Algorithm \ref{algo:streaming-modular} is compliant with parameter $\gamma$. By taking $\gamma = 1$, we obtain $\frac{1}{4p}$-approximation for $p$ matroids intersection and $1/4$-approximation for matroid constraint.










\subsection{Results for Matching and $p$-matchoid}
For matching constraint, Chakrabarti et al. \cite{CK14} noted that the algorithm in \cite{Z12} for streaming graph matching is a compliant algorithm. By applying Theorem \ref{thm:compliant}, they obtained an algorithm for normalized monotone submodular with $\frac{4}{31}$-approximation guarantee.

\cite{CGQ15} is the only paper that gives some results on $p$-matchoid. We will not discuss its technical details but summarize its results in Table \ref{table:streaming}.